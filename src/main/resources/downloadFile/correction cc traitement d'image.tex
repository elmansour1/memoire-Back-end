\documentclass{article}
\usepackage[utf8]{inputenc}
\usepackage{amsmath}
\usepackage{amssymb}
\usepackage{mathrsfs}
\usepackage{anysize}
\usepackage{fancyhdr}
\pagestyle{fancy}
\renewcommand\headrulewidth{1pt}
\fancyhead[L]{Uma - ENSPM}
\fancyhead[R]{INFOTEL - Niveau 5}
\renewcommand\footrulewidth{1pt}
\fancyfoot[L]{Gelo-Reseaux}
\fancyfoot[C]{\today}
\fancyfoot[R]{faelma}
\marginsize{15mm}{10mm}{10mm}{10mm}
\begin{document}
\begin{center}\textbf{Correction CC Traitement d'image 2018-2019}
\end{center}
\paragraph{Exercice 1:\\}
\begin{itemize}
\item[1]. Définition:\\
Image: Une application d'un sous-ensemble de (MxN) de RxR vers l'ensemble des réels R qui a chaque couple des réels(x,y) associe un réel f(x,y).\\
Image Numérique: un signal fini bidimensionnel échantillonné à valeur quantifiée dans un certain espace de couleurs.\\
Histogramme:Courbe statistique indiquant la répartition des dessins selon leurs valeurs.\\
Image binaire: une image (MxN) où chaque point peut prendre uniquement la valeur 0 ou 1. 
\item[2].La différence est simple. Dans une image au format RAW, l’entête et les données sont dans deux fichiers différents alors que dans une image au format BMP, l’entête et les données sont dans un seul fichier.
\item[3]. Une image mono-bande est une image prise dans une seule bande spectrale alors qu'une image multibande est prise dans plusieurs bandes spectrale.
\item[4].Complétons le tableau suivant: \\
\begin{tabular}{|c|c|}
\hline
Morphologie mathématique & fonction'\\
\hline
érosion & Opération ou une fonction qui consiste à réduire l'entité\\
\hline
dilatation & Opération ou une fonction qui consiste à augmenter la taille de l'entité \\
\hline
Squelettisation &  Opération ou une fonction qui consiste à réduire l' épaisseur de l'image\\
\hline
filtrage & Opération qui consiste à éliminer les bruits sur une image avant de la traitée\\
\hline
binarisation & Opération ou une fonction qui consiste à remplacer les gris d'une image par 0 ou 1\\
\hline
\end{tabular}
\end{itemize}

\paragraph{Exercice 2:\\}
soit l'image suivante:
\begin{tabular}{|c|c|c|}
\hline
0 & 4 & 3 \\
\hline
1 & 3 & 0 \\
\hline
2 & 0 & 3 \\
\hline 
\end{tabular} et le masque 
\begin{equation*}
M=
\begin{pmatrix}
1 & 1 & 1\\
1 & 1 & 1\\
1 & 1 & 1
\end{pmatrix}
\end{equation*}
En appliquant la methode de bordure zero, nous aurons: \begin{tabular}{|c|c|c|c|c|}
\hline
0 & 0 & 0 & 0 & 0 \\
\hline 
0 & 0 & 4 & 3 & 0 \\
\hline
0 & 1 & 3 & 0 & 0 \\
\hline
0 & 2 & 0 & 3 & 0 \\
\hline 
0 & 0 & 0 & 0 & 0 \\
\hline 
\end{tabular}\\ \\
 C1 = (0*1)+(0*1)+(0*1)+(0*1)+(0*1)+(1*4)+(0*1)+(1*1)+(3*1) = 8 \\
 NG1 = E(0*8/4) = 0 \\ \\
 C2 = (0*1)+(0*1)+(0*1)+(0*1)+(0*1)+(1*4)+(3*1)+(1*1)+(3*1) = 11\\
 NG2 = E(4*11/4) = 11 \\ \\
 C3 = (0*1)+(0*1)+(0*1)+(4*1)+(3*1)+(1*0)+(3*1)+(0*1)+(0*1) = 10 \\
 NG3 = E(3*10/4) = E(7.5) = 7 \\ \\
 C4 = (0*1)+(0*1)+(4*1)+(0*1)+(1*1)+(1*3)+(0*1)+(2*1)+(0*1) = 10 \\
 NG4 = E(1*10/4) = E(2.5) = 2\\ \\
 C5 = (0*1)+(4*1)+(3*1)+(1*1)+(1*3)+(0*1)+(2*1)+(0*1)+(3*1) = 16 \\
 NG5 = E(3*16/4) = 12 \\ \\
 C6 = (4*1)+(3*1)+(0*1)+(1*3)+(0*1)+(0*1)+(0*1)+(3*1)+(0*1) = 13 \\
 NG6 = E(0*13/4) = 0 \\ \\ 
 C7 = (0*1)+(1*1)+(3*1)+(0*1)+(2*1)+(0*1)+(0*1)+(0*1)+(0*1) = 6 \\
 NG7 = E(2*6/4) = 3 \\ \\
 C8 = (1*1)+(3*1)+(0*1)+(2*1)+(0*1)+(3*1)+(0*1)+(0*1)+(0*1) = 9 \\
 NG8 = E(0*9/4) = 0 \\ \\
 C9 = (3*1)+(0*1)+(0*1)+(0*1)+(3*1)+(0*1)+(0*1)+(0*1)+(0*1) = 6 \\
 NG9 = E(3*9/4) = E(4.5) = 4 \\ \\
 
 L'image finale après la convolution est
 \begin{tabular}{|c|c|c|}
\hline
0 & 11 & 7 \\
\hline
2 & 12 & 0 \\
\hline
3 & 0 & 4 \\
\hline 
\end{tabular}
\end{document}